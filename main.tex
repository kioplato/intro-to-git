\documentclass[12pt,a4]{article}
\usepackage[greek,english]{babel}
\usepackage{graphicx}
\graphicspath{ {./images/} }
\usepackage{coffee4}
\usepackage{hyperref}

\textwidth      6.5in
\textheight     8.7in
\oddsidemargin  0.0cm
\evensidemargin 0.0cm
\parindent      0pt
\topmargin      0.1in
\headsep        20pt

\parskip        10pt
\parindent      0pt

\newcommand{\gr}{\greektext}
\newcommand{\la}{\latintext}

\newcommand{\ilcode}[1]{\textcolor[RGB]{160, 110, 220}{#1}}

\begin{document}

\begin{titlepage}
	\begin{center}
		% Title.
		\vspace*{1cm}
		\Huge
		\textbf{{\la Intro to Git}}
		
		% Subtitle.
		\vspace{0.5cm}
		\LARGE
		{\sf \la Workshop}

		% Participants in alphabetical order.
		\vspace{1.5cm}
		\large
		\textbf{\la Kiorpelidis Platon}\\
		\textbf{\la Baziotis Stefanos}

		\vfill
		
		%\includegraphics[width=0.4\textwidth]{uni-logo}
		\scalebox{1.5}{%
			\hspace*{0.43cm}\includegraphics[width=0.4\textwidth]{acmuoa}}

		\LARGE
		{\sf \la University of Athens}\\
		{\sf \la ACM UoA Student Chapter}
	\end{center}
\end{titlepage}


\section{Introduction}
\subsection{Git \& Version Control}
{\sf -- \emph{The Problem}:}
Many people’s version-control method of choice is to copy files into another
directory. We create copies of the project and name them to \emph{project v2},
\emph{project v3}, etc. This approach is very common because it is so simple,
but it is also incredibly error prone. It is easy to forget which directory
you’re in and accidentally write to the wrong file or copy over files you don’t
mean to -- whenever you have the entire history of the project in a single
place, you risk losing everything. In addition, it clutters up the workspace and
gives vertigo to anyone browsing it.

We also want to collaborate with other developers. We could keep the codebase
under a server but what about server failures. If the server goes down for an
hour, then during that hour nobody can collaborate at all or save versioned
changes to anything they're working on. If the hard disk the central database is
on becomes corrupted, and proper backups haven’t been kept, you lose absolutely
everything -- the entire history of the project except whatever single snapshots
people happen to have on their local machines.

{\sf -- \emph{The Solution}:} This is where Distributed Version Control Systems
step in, like Git. In such systems, clients don’t just check out the latest
snapshot of the files; rather, they fully mirror the repository, including its
full history. Thus, if any server dies, and these systems were collaborating via
that server, any of the client repositories can be copied back up to the server
to restore it. Every clone is really a full backup of all the data.

\textcolor[RGB]{220,220,220}{\rule{\linewidth}{0.8pt}}

{\sf Version Control:} It's is a system that records changes to a file or a set
of files over time so that you can recall specific versions later.

{\sf Distributed Version Control System:} Clients fully mirror the remote
repository, including it's history. Remote repository is stored in a server.
Full copy of the code is present in all the developers' computers.

Popularity of Git (2010 vs 2019):
\vspace*{-5pt}
\begin{itemize}
\item 2010: 26,485 repositories (11.3\% of total)
\item 2019: 913,378 repositories (70\% of total)
\end{itemize}

{\sf Server example:} GitHub. GitHub is the single largest hosting server for
Git repositories with a web front end. It offers all of the distributed version
control and source code management (SCM) functionality of Git as well as adding
its own features. It provides access control (users) and several collaboration
features such as bug tracking, feature requests, task management, and wikis for
every project. A large percentage of all Git repositories are hosted on GitHub,
and many open-source projects use it for Git hosting, issue tracking, code
review, and other things. So while it’s not a direct part of the Git open source
project, there’s a good chance that you’ll want or need to interact with GitHub
at some point while using Git professionally.

\subsection{Concept of Git \& Early days}
Git follows the unix philosophy. The Unix philosophy emphasizes building simple,
short, clear, modular, and extensible code that can be easily maintained and
repurposed by developers other than its creators. The Unix philosophy favors
composability as opposed to monolithic design.

In it's early days git consisted of simple, atomic executables. Such programs
weren't enough to be useful on their own, but a combination of those, usually in
shell scripts, composed more complex functionality like \ilcode{\bf git add}.
Probably it's the reason, after so many developement cycles, still considered a
complex and difficult tool.

It was created by Linus Torvalds to help maintain Linux. As more people were
sending in patches and the current workflow didn't scale, Linus created a new
VCS as he didn't like existing ones. He named it git -- the stupid content
tracker or the information manager from hell.

\subsubsection{Meaning of Git}
\vspace*{-10pt}
\emph{Copied from the first git commit written by Linus Torvalds.}

"git" can mean anything, depending on your mood.
\begin{itemize}
\vspace*{-10pt}
\item Random three-letter combination that is pronounceable, and not actually
	used by any common UNIX command. The fact that it is a mispronounciation of
	"get" may or may not be relevant.
\vspace*{-3pt}
\item Stupid. Contemptible and despicable. Simple. Take your pick from the
	dictionary of slang.
\vspace*{-3pt}
\item "global information tracker": you're in a good mood, and it actually
   works for you. Angels sing, and a light suddenly fills the room.
\vspace*{-3pt}
\item "goddamn idiotic truckload of sh*t": when it breaks.
\vspace*{-10pt}
\end{itemize}

This is a stupid (but extremely fast) directory content manager. It doesn't do
a whole lot, but what it does do is track directory contents efficiently.

\subsection{Modern Git}
Initially git wasn't developed for the novice user. Nowdays it's a lot
friendlier and easier to get into as it comes with a lot of build-in commands,
wrappers and complexity abstractions. These commands internally use the old,
enhanced, but still atomic executables. There are GUIs developed for the GUI
lovers or non-technical people and it's supported with plugins by various text
editors and IDEs. These changes and features make for an easier, faster and
streamlined collaboration between developers, who follow various local and
online workflows.

\section{Installing Git}
Before you start using Git, you have to make it available on your computer. Even
if it’s already installed, it’s probably a good idea to update to the latest
version. You can either install it as a package or via another installer, or
download the source code and compile it yourself.

{\sf -- Linux (Debian, Ubuntu):} \ilcode{sudo apt-get install git-all}

{\sf -- Max:} Run \ilcode{git -{}-version}. If you don’t have it installed
already, it will prompt you to install.

{\sf -- Windows:} Download from \url{https://git-scm.com/download/win}.
\end{document}
